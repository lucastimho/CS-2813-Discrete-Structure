\documentclass{article} % This command is used to set the type of document you are working on such as an article, book, or presentation

\usepackage{geometry} % This package allows the editing of the page layout
\usepackage{amsmath}  % This package allows the use of a large range of mathematical formula, commands, and symbols
\usepackage{graphicx}  % This package allows the importing of images
\usepackage{enumerate} %This package allows the use of making lists

\newcommand{\question}[2][]{\begin{flushleft}
        \textbf{Question #1}: \textit{#2}

\end{flushleft}}
\newcommand{\sol}{\textbf{Solution}:} %Use if you want a boldface solution line
\newcommand{\maketitletwo}[2][]{\begin{center}
        \Large{\textbf{Assignment 1}
            
            CS 2813 - Discrete Structures} % Name of course here
        \vspace{5pt}
        
        \normalsize{Lucas Ho  % Your name here
        
        September 4, 2023}        % Due date
        \vspace{15pt}
        
\end{center}}
\begin{document}
    \maketitletwo[5]  % Optional argument is assignment number
    %Keep a blank space between maketitletwo and \question[1]
    
    \question[0]{This book[1] has the the answer to everything. What is the question?} 

    What is the meaning of life?
    
    \question[1]{The propositions p NAND q and p NOR q are denoted by p $\mid$ q and p $\downarrow$ q, respectively.}
    
    \begin{enumerate}[a.]
        \item {Construct a truth table for the logical operators NAND and NOR.}
        
        \begin{center}
            \begin{tabular}{|c|c|c|c|}
                \hline
                p & q & p $\mid$ q & p $\downarrow$ q \\
                \hline
                T & T & F & F \\
                T & F & T & F \\
                F & T & T & F \\
                F & F & T & T \\
                \hline
            \end{tabular}

        \end{center}

        \item {Show that p $\mid$ q is logically equivalent to $\neg$(p $\land$ q).}
        
        \begin{center}
            \begin{tabular}{|c|c|c|c|}
                \hline
                p & q & $\neg$(p $\land$ q) & p $\mid$ q \\
                \hline
                T & T & F & F \\
                T & F & T & T \\
                F & T & T & T \\
                F & F & T & T \\
                \hline
            \end{tabular}

        \end{center}
    \end{enumerate}
    
    \question[2]{A proposition is satisfiable if some setting of the variables makes the proposition true. For example, p $\land$ $\neg$q is satisfiable because the expression is true if p is true and q is false. On the other hand, p $\land$ $\neg$p is not satisfiable because the expression as a whole is false for both settings of p. But determining whether or not a more complicated proposition is satisfiable is not so easy. How about this one?}

    \begin{center}
        (p $\lor$ q $\lor$ r) $\land$ ($\neg$p $\lor$ $\neg$q) $\land$ ($\neg$p $\lor$ $\neg$r) $\land$ ($\neg$r $\lor$ $\neg$q)
    \end{center}

    \begin{enumerate}[a.]
        \item {Find whether or not the previous compound proposition is satisfiable?}
        
        \begin{center}
            \begin{tabular}{|c|c|c|c|c|c|c|}
                \hline
                p & q & r & p $\lor$ q $\lor$ r & $\neg$p $\lor$ $\neg$q & $\neg$p $\lor$ $\neg$r & $\neg$r $\lor$ $\neg$q \\
                \hline
                T & F & F & T & T & T & T \\
                \hline
            \end{tabular}

        \end{center}

        The previous compound proposition is satisfiable.

        \item {Given the following disjunctions, determine whether each is satisfiable. Use a different approach to how
        you solved a).}
        \begin{enumerate}[i)]
            \item {(p $\lor$ q $\lor$ $\neg$r) $\land$ (p $\lor$ $\neg$q $\lor$ $\neg$s) $\land$ (p $\lor$ $\neg$r $\lor$ $\neg$s) $\land$ ($\neg$p $\lor$ $\neg$q $\lor$ $\neg$s) $\land$ (p $\lor$ q $\lor$ $\neg$s)}
            
            This proposition is satisfiable for any p or q as long as r and s are false.

            \item {($\neg$p $\lor$ $\neg$q $\lor$ r) $\land$ ($\neg$p $\lor$ q $\lor$ $\neg$s) $\land$ (p $\lor$ $\neg$q $\lor$ $\neg$s) $\land$ ($\neg$p $\lor$ $\neg$r $\lor$ $\neg$s) $\land$ (p $\lor$ q $\lor$ $\neg$r) $\land$ (p $\lor$ $\neg$r $\lor$ $\neg$s)}
            
            This proposition is not satisfiable as it is not possible for both ($\neg$p $\lor$ $\neg$q $\lor$ r) and (p $\lor$ q $\lor$ $\neg$r) to both occur.

            \item {(p $\lor$ q $\lor$ r) $\land$ (p $\lor$ $\neg$q $\lor$ $\neg$s) $\land$ (q  $\lor$ $\neg$r $\lor$ s) $\land$ ($\neg$p $\lor$ r $\lor$ s) $\land$ ($\neg$p $\lor$ q $\lor$ $\neg$s) $\land$ (p $\lor$ $\neg$q $\lor$ $\neg$r) $\land$ ($\neg$p $\lor$ $\neg$q $\lor$ s) $\land$ ($\neg$p $\lor$ $\neg$r $\lor$ $\neg$s)}
            
            This proposition is not satisfiable because p is unable to be both true and false. p is of note here because after expanding the compound proposition with logical equivalences, there are propositions of (p $\land$ $\neg$p) which help determine satiability.
        \end{enumerate}
        \item {Alice claims she wrote an ’efficient’ algorithm for determining whether a compound proposition is satisfiable or not. Bob believes Alice; can Bob use Alice’s algorithm to determine whether a compound
        proposition p is a tautology or not?}

        Yes 

        \item {Why is it not easy to generalize a truth table for any complex proposition? In other words, is there a
        more efficient solution to this problem? Do you think Alice from c) is lying?}

        You can use logical equivalences to solve any complex propositions so Alice is not lying.

    \end{enumerate}

    \question[3] {Show that $\neg$p $\rightarrow$ (q $\rightarrow$ r) and q $\rightarrow$ (p $\lor$ r) are logically equivalent. Use Logical Equivalences not truth tables.}

    p $\lor$ (q $\rightarrow$ r)

    p $\lor$ $\neg$q $\lor$ r

    $\neg$q $\lor$ p $\lor$ r

    q $\rightarrow$ (p $\lor$ r)

    \begin{center}
        REFERENCES
    \end{center}

    [1] D. Adams. \emph{The Hitchhiker’s Guide to the Galaxy.} San Val, 1995.
\end{document}